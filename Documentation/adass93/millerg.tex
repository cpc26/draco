% DRACO paper for October 1993 ADASS Conference in Victoria, BC
% Draco: An Expert Assistant for Data Reduction and Analysis
% Glenn Miller and Felix Yen

\documentstyle[11pt,paspconf]{article}

\begin{document}

\title{DRACO: An Expert Assistant for Data Reduction and Analysis}
\author{Glenn Miller and Felix Yen}
\affil{Space Telescope Science Institute, 3700 San Martin Dr., Baltimore MD 21218}

\begin{abstract}

The use of large format detectors, increased access to very large 
astronomical databases, and other developments in observational astronomy 
have led to the situation where many astronomers are overwhelmed by the 
reduction and analysis process. This paper reports a novel approach to data 
reduction and analysis which works in conjunction with existing analysis 
systems such as {\sc stsdas/iraf}. This system, called {\sc draco}, takes 
on much of the mechanics of the process, allowing the astronomer to spend 
more time understanding the physical nature of the data. In developing {\sc 
draco} we encountered a number of shortcomings of current data analysis 
systems which hinder the ability to effectively automate routine tasks. We 
maintain that these difficulties are fundamental (not specific to our 
approach) and must be addressed by conventional and next-generation 
analysis systems. 

\end{abstract}

\keywords{data reduction, data analysis, data management, user interface, 
networking, interoperability, expert assistant}

\section{Introduction}

The task of data reduction presents severe obstacles to an astronomer: The 
volume of data may require tedious work that is susceptible to errors 
(e.g., the flat-fielding and bias correction of a few dozen digital images 
can take several day's time and it is easy to accidentally apply the wrong 
calibrations to some of the images). Management of the data reduction 
process may require tracking tens or hundreds of files through many 
different steps. The quality of each reduction step should be evaluated 
(e.g., stability of internal calibrations, or abnormally large number of 
bad pixels). Often the entire reduction process must be repeated several 
times with improved calibration data or improved reduction algorithms.

These are significant problems that inhibit progress by forcing the 
scientist to expend time and effort on the mechanics of reduction rather 
than understanding the physical nature of the data. We have developed {\sc 
draco}, which is a tool for the management of data reduction and analysis. 
{\sc draco} builds on the foundation of existing data analysis systems such 
as {\sc stsdas/iraf, idl, midas,} etc. {\sc draco} gathers information 
about the available data, develops a plan for data reduction based on a 
template supplied by the astronomer, and translates the plan into explicit 
reduction commands. An important feature of {\sc draco} is its generality 
and extensibility - new types of data analysis tasks or additional data 
analysis systems can easily be added without modifying existing software. 
This work is an extension of a successful prototype system for the 
calibration of CCD images developed by Johnston (1987). 

\section{The Draco System}

To use {\sc draco}, the astronomer first describes the reduction process by 
defining three key entities:
\begin{itemize}

 \item {\it Procedure} - This is an abstract user program for the 
reduction, e.g. bias correction, dark removal, field flattening, and 
extraction of a spectrum
	
	\item  {\it Primitive} - These are the abstract data analysis operations 
	used to build a procedure (e.g. bias correction or flat fielding).
	
	\item  {\it Implementation} - These implement primitives, usually by 
	invoking an underlying analysis system, e.g. {\sc iraf, stsdas}, etc.
			
\end{itemize}

Primitives are the basic building blocks which are used to construct the 
reduction procedures and insulate the astronomer from many of the details 
of the underlying analysis systems. Primitives form a library of routines 
from which new reduction and analyses can be adapted from existing ones. 
Adding a new analysis package to {\sc draco} consists of creating the 
appropriate implementations for a set of primitives. Refer to Miller 
(1992), Yen (1993) for more details on Draco.

Once these entities are defined, the astronomer invokes {\sc draco}'s {\tt 
start} function, specifying the directory containing the data. {\sc draco} 
gathers information about the data at hand (usually by reading header 
information in the files), expands the procedure into a reduction plan 
based on the actual data, creates a command language script in the target 
analysis system and finally executes the script. The time-consuming 
reduction takes place without further attention from the astronomer. {\sc 
draco} logs all steps for later review and calls attention to problems 
such as missing data or calibration files.

By producing a command script in the language of a data analysis system, 
{\sc draco} builds on the foundation of these systems, rather than creating 
yet another analysis system. It is common for astronomers to write 
operating system command language scripts (e.g., Unix Shell or VMS DCL) to 
reduce data. The advantages of the {\sc draco}-generated scripts are clear: 
{\sc draco} provides a higher level of abstraction and handles many lower 
level details for the user. It is usually very difficult to modify custom 
command language scripts for different reduction tasks whereas {\sc draco} 
facilitates reuse of its component data structures. 

\subsection{Experience}

{\sc draco} has been applied to two separate astronomical projects with 
successful results. The first version of {\sc draco} was used to manage the 
removal of cosmic ray artifacts from a sample of HST WF/PC data for the 
Medium-Deep Survey Key Project (provided by Griffiths and Ratnatunga). A 
revised version of {\sc draco} was used to extract spectroscopic data from 
ground-based telescope data (provided by MacConnell and Roberts), including 
bias and dark removal, flattening and extraction of the spectra. In this 
latter case, {\sc draco} managed a significant amount of data: ~325Mb in 
~650 separate files. 

\subsection{Availability}

{\sc draco} is an initial system which addresses the issues of automating 
data reduction and analysis. In order to encourage development of similar 
ideas and systems, {\sc draco} (including source code and documentation) is 
available to the community from the Space Telescope Science Institute 
server ({\tt stsci.edu}) via anonymous ftp, Gopher and World Wide Web. 

\section{Discussion}

In the course of this project we had extensive discussions with many 
research groups. Contrary to the prevailing view that the lack of 
visualization tools or graphical user interfaces is a major impediment to 
research, we found that a more serious problem for all groups was the 
difficulty of managing the data reduction process. {\sc draco} demonstrates 
a simple and effective way to perform data reduction with less human 
interaction and fewer errors. Other approaches are being explored such as 
the Khoros system (Rasure 1991) and commercial systems such as Silicon 
Graphics Explorer or BBN Cornerstone. We encourage developers of 
astronomical data analysis systems to incorporate these ideas into 
future work. From our experience, we can identify a number of general 
capabilities which would greatly enable astronomical researchers. Users 
need tools to: 

\begin{itemize} 

\item Describe the data at hand - Directory listings of files are usually 
the only tool available to describe the data. Users need more powerful 
tools (e.g. graphical) which understand and display the types of data (flat, bias, comparison 
spectrum, etc.) and the relationships between the data (e.g. identifying 
multiple images of the same target).

\item Describe reduction process and algorithms at a high level - Users 
generally have to work at quite a low level, dealing with the specifics 
of the analysis system, operating system, file types, etc.

\item Facilitate experimentation with reduction parameters and different 
algorithms - Data reduction is an integral part of the scientific process 
and it is therefore vital to experiment with relevant parameters of the 
reduction (number of iterations in a restoration algorithm, cosmic ray 
removal parameters, etc.). If reducing the data just once requires too 
much time and effort then experimentation becomes impractical and an 
important aspect of the scientific method is sacrificed.

\item Make it possible to resume data reduction/analysis after 
interruptions.

\item Perform data quality checks.

\item Provide traceability: Too often data analysis systems do not to adequately 
document what operations were performed on the data. As a minimum, systems 
must document in the headers what steps were performed, including input 
data, algorithm, parameters and software version. 
\end{itemize} 

A change in the basic philosophy of scientific data analysis systems is 
needed. Rather than being designed as monolithic systems which are the 
complete environment for all data reduction and analysis, systems should 
begin a migration towards ``interoperability'' where they can be invoked by 
other systems and software. A client-server technology (communicating via 
Unix sockets or some other protocall) seen in many other computer 
applications seems a likely architecture to fullfil this goal. 

\section{Summary}

Management of the data reduction process is an important problem facing 
astronomers who deal with observational data. The lack of effective data 
management tools can often be overwhelming. 
{\sc draco} demonstrates one approach to providing automated assistance 
in order to free the astronomer to concentrate on scientific issues. {\sc 
draco} works in concert with existing data analysis systems.
We assert that current and future data analysis systems must provide 
tools for effective automation of procedures and data management.


\acknowledgments

We thank Mark Johnston, Bob Hanisch, Ron Gilliland, Richard Griffiths, 
Keith Horne, Jack MacConnell, Jim Roberts, Kavan Ratnatunga and Phil Martel 
for discussions about data reduction and their thoughtful comments on the 
design and development of {\sc draco}. This work is supported by NASA's 
Astrophysics Information Systems Research Program through CESDIS by a 
contract with the Space Telescope Science Institute which is operated by 
AURA for NASA. 


\begin{references}

\reference Johnston, M. 1987, ``An Expert System Approach to Astronomical 
Data Analysis'', Proceedings of the Goddard Conference on Space Applications 
of Artificial Intelligence, NASA. 

\reference Miller, G. 1992, ``The Data Reduction Expert Assistant'', 
Astronomy from Large Databases II, Hagenau, France, ed. A. Heck and F. 
Murtagh, ESO.

\reference Rasure, J. and M. Williams 1991, ``An Integrated Visual Language 
and Software Development Environment'', Journal of Visual Languages and 
Computing, 2: 217-246.

\reference Yen, F. 1993 ``Draco Design Document'', STScI APSB Technical 
Report 1992-07, available from {\tt stsci.edu}.


\end{references}

\end{document}
